\section{Conclusions}\label{sec-conclusions}
Experimental researchers know that experimentation, as described in textbooks and articles, is only an approximation to the daily work in the labs and the field. This study shows the deviations found in a concrete experimental research group. Such divergences do not seem arbitrary but necessary from a systemic viewpoint, e.g., each people have distinct abilities and perform different tasks in a research project.

We do not claim that our findings are commonplace in the empirical community. They may be particular to the experimental research group in which we performed the ethnographic study. However, some clues evidence that peculiar behaviors are usual in the labs and the field. For instance, natural sciences researchers have complained about the replication problem \cite{hines2014sorting}. At least in the sciences (maybe not in SE), replication frequently fails due to experimental setup and conduction changes. These changes are motivated by the researchers' tacit knowledge \cite{Polanyi-1996-tacit-k} \cite{Shull-2002-replicating-SE-experiments-tacit-k}, and consequently, protocols or experiment reports do not declare such changes. Experiment variations happen so frequently that some disciplines coined a (comic) term: \textquotedblleft lab mythology \textquotedblright~\cite{ruben2011experimental} \cite{loukides2015beyond}. Our study has shown that such tacit knowledge is present in SE and manifests at a relatively superficial level, e.g., when researchers from different experimental families discuss their mental models.

Furthermore, anecdotic evidence, e.g., conversations with colleagues, suggests that this "lab mythology" can also be somewhat present in SE. We all know that the experiment reports do not reflect the experimental process precisely. It does not mean experimenters cheat. We all know from experience that some practices work and others do not, and we apply them as required.

We believe that the "lab mythology" is worth studying. If the differential characteristics of the theoretical vs. actual experiment process are research-group specific, the inquiry will end. However, it is also possible that widespread patterns of behavior show up. Bringing such knowledge to the foreground could contribute to the improvement of experimental practice.

We are aware that the findings of this study correspond to the particular perspective of an experimental research group, albeit representative of the ESE community. Therefore, it is necessary to carry out further studies to generalize the results of the present work. We have completed a survey exploring some issues, e.g., terminological diversity, that surfaced during the ethnography. We have also completed a comparative study of the experimental process between the SE and an established engineering discipline. Both works will likely be published shortly. In addition, we plan to conduct more studies (probably not ethnographic, because they require too much time) on this topic soon.