\section{Conclusions}\label{sec-conclusions}
Experimental researchers know that experimentation, as described in textbooks and articles, only approximates the daily work in the lab and the field. This study shows the deviations found in a concrete experimental research group. Such divergences do not seem arbitrary but are necessary from a systemic viewpoint, e.g., people have distinct abilities and perform different tasks in a research project.

We do not claim that our findings are commonplace in the empirical community. They may be particular to the experimental research group where we performed the ethnographic study. However, some clues evidence that peculiar behaviors are usual in the lab and the field. For instance, natural sciences researchers have complained about the replication problem \cite{hines2014sorting}. At least in the sciences (maybe not in SE), replication frequently fails due to experimental setup and conduction changes. These changes are motivated by the researchers' tacit knowledge \cite{Polanyi-1996-tacit-k,Shull-2002-replicating-SE-experiments-tacit-k}, and consequently, protocols or experiment reports do not declare such changes. Experiment variations happen so frequently that some disciplines coined a (comic) term: \textquotedblleft lab mythology \textquotedblright~\cite{ruben2011experimental,loukides2015beyond}. Our study has shown that such tacit knowledge is present in SE and manifests superficially, e.g., when researchers from different experimental families discuss their mental models.

Furthermore, anecdotic evidence, e.g., conversations with colleagues, suggests that this "lab mythology" can also be somewhat present in SE. We all know that the experiment reports do not reflect the experimental process precisely. It does not mean experimenters cheat. We all know from experience that some practices work and others do not, and we apply them as required.

\subsection{Implications for practice}

\added[id=v3]{Our findings suggest specific improvements that could be applied to SE experimentation. The first and most evident is transitioning from an experimental process primarily based on the application of experimental designs and the conduct of statistical analyses to a more professional approach based on explicit processes, activities, and roles, as is customary in many areas, e.g., Information Technology \cite{iso20000}.}

\added[id=v3]{It could be argued that current SE experimentation already has a well-defined process and activities, as can be seen, e.g., in the works of Juristo \& Moreno \cite{Juristo-2001-SE-experimentation} or Wohlin \textit{et al.} \cite{Wohlin-2000-Experimentation-SE}. However, as we have argued before, such a process and activities represent only an idealization of the activities performed in practice. We refer to more detailed processes and activities (such as the process models we provide in this work).}

\added[id=v3]{Detailed processes and activities only make sense within specific research areas or even families of experiments. For example, Arcuri \& Briand \cite{Arcuri-2014-randomized} have published recommendations for the statistical analysis of randomized algorithms. Arcuri \& Briand do not claim that their recommendations have general applicability. Other areas of SE, such as programming or testing, will need specific recommendations. Although this may seem strange or unusual, a cursory examination of experimental literature in other sciences demonstrates the opposite. For example, in Psychology, there are recommendations on how to measure motivation \cite{toure2014measure} or consider experience \cite{hoffman1998can}. In Medicine, there are models of pain \cite{petersen2002pain}. Other examples could be cited, but all of them agree on highlighting the specificities of a research area instead of hiding details in abstractions that ultimately prove inoperative.}

\added[id=v3]{Finally, we must add specific competencies to processes, activities, and roles, such as negotiation skills, asset curation, or human resource management. Again, SE is lagging behind established trends in other areas. To cite a very relevant example, the European Union \cite{ResearchComp} has defined a framework containing seven competency areas, 38 competencies, and 389 learning outcomes for researchers. Other institutions have undertaken similar efforts.}

\subsection{Limitations}

\added[id=v2]{However, this work is subject to various limitations. The first and most important is the empirical basis obtained. Although ethnographic research has been extensive over time and conducted using multiple techniques, in the end, we only have data from one research group. Furthermore, this group's experimenters have self-trained in the field, meaning they have not received specific academic training. The observed peculiarities could be habits acquired over time that other, more recent groups with formal training do not possess.}

\added[id=v2]{We also cannot rule out errors in perception on the researchers' part as well. As mentioned earlier, we had yet to learn of experimentation at the beginning of ethnographic research, which may have led us to misinterpret some behaviors of the experimenters. On the other hand, perhaps the researchers' "fresh" perspective has allowed us to identify our findings.}

\subsection{Future work}

\deleted[id=v2]{We believe that the ''lab mythology'' is worth studying. If the differential characteristics of the theoretical vs. actual experiment process are specific to the research group, the inquiry will simply end. However, it is also possible that widespread patterns of behavior show up. Bringing such knowledge to the foreground could contribute to improving experimental practice.}

\deleted[id=v2]{We are aware that the findings of this study correspond to the particular perspective of an experimental research group, albeit representative of the ESE community. Therefore,}
\added[id=v2]{To overcome the limitations of the present investigation,} it is necessary to carry out further studies to generalize the results of the current work. \added[id=v2]{Our plans are as follows:}
\begin{itemize}

\item \added[id=v2]{Verify the existence of terminological diversity, one of the first aspects that have emerged in the research and is relatively easy to check. We have already started this work. We surveyed SE experimenters from multiple research groups. In the survey, we asked them about the meaning and usage of experimentation-related terms. Preliminary results confirm the existence of terminological diversity.}

\item \added[id=v2]{After completing the ethnographic research, we initiated a comparative study of the experimental process between SE and an established engineering discipline (specifically, biotechnology). Preliminary results indicate the absence of terminological diversity and a greater formalization of the experimental process within specific research areas but not at a general level. This could be the key to terminological diversity in SE: sub-areas of SE generally have not developed specific terminology and protocols which could lead to the need for ad-hoc concepts and processes.}

\end{itemize}

\replaced[id=v2]{Both works will likely be published shortly. In addition, we plan}{We plan} to conduct more studies \replaced[id=v2]{in the future}{(probably not ethnographic because they require too much time)} on this topic soon. \added[id=v2]{We still need to decide on the objects under study and the methodologies used since ethnography is very demanding. A possibility to study the impact of specialization in the sciences is to analyze experimental repositories. This represents an indirect but efficient approach in terms of time and effort. Another alternative is conducting qualitative studies using focus groups composed of SE experimental researchers.}

