\section{Introduction}\label{sec-introduction}
Replication is troublesome in general \cite{Klein-2018-many}, particularly in the social and life sciences \cite{Pashler-2012-perspectives,Baker-2016-lid-reproducibility}. In Empirical Software Engineering (ESE%
\footnote{\added[id=v4]{The acronyms used throughout the paper are listed in Table~\ref{tab:acronyms} in the appendix.}}%
)\added[id=v3]{, there is a consensus that replications are important \cite{shepperd2018role}, but} the number of replications is small \cite{Bezerra-2015-Replication-SE-U-SMS} and the rate of confirmation of previous results is limited \cite{Jorgensen-2016-Incorrects-Results-SEE}.

Several reasons prevent replication. They have discussed elsewhere, e.g., \cite{Miller-2005-replicating-SE-experiments,Demagalhaes-2015-replications-SE,cockburn2020threats,mahmood2018reproducibility,dos2022investigating}. In this paper, we focus on a potential cause that, to our knowledge, has not been explored so far: How ESE researchers \textit{really} experiment. 

Anecdotic evidence such as conversations with other researchers, review of articles, and direct observation suggests that researchers follow \textit{in practice} different \textit{experimental processes}, i.e., the different activities that ESE researchers perform to conduct experiments. Experiment reports look relatively uniform due to the existence of reporting guidelines \cite{Carver-2010-guidelines-replication-SE,Jedlitschka-2008-reporting-experiments-SE}, but the underlying data generation processes vary.

This paper aims to \replaced[id=v2]{investigate}{determine} \textbf{how experimental researchers conduct experiments in practice}. \added[id=v2]{We intend to find out how experimental researchers plan, execute, analyze, and report their investigations and what concepts, protocols, and processes they use.}

We applied an ethnographic method \cite{Sharp-2016-Ethnographic-Studies-ESE,zhang2019ethnographic} within an experienced experimental research group. We observed the researchers in their lab for two years. We created conceptual and process models to represent the concepts they use and the activities they perform daily. These models fit the community's procedures and terminology at a high level. Still, the models show particularities at lower levels: (1) The number and diversity of activities, (2) the existence of different roles, (3) the granularity of the concepts used by the roles, and (4) the viewpoints that different subareas or families of experiments have about the overall process.

The contributions of this paper are:

\begin{itemize}
  \item Identifying activities that depart from or are not described in mainstream experimentation procedures. 
  \item Characterizing roles, i.e., groups of experimental activities performed by the same person, in ESE research.
  \item Evidence of the existence of tacit knowledge in ESE, which, in turn, leads to discrepancies between experimentation protocols and actual practice.
  \item Conceptual and process models that represent how experimentation is conducted \textit{in practice} in an experimental research group.
\end{itemize}

Finally, we wish to open a discussion in the ESE community: 
\begin{itemize}
  \item Should those differential characteristics in experimentation in software engineering be disregarded as inherently contextual?
  \item  \textbf{Alternatively}: Are they more common than they may seem, so they should be further investigated and eventually integrated into mainstream practice?
\end{itemize}

The remainder of the article is structured as follows: Section \ref{sec-research-method} describes the research method. Section \ref{sec-reseach-execution} reports the stages of the ethnographic research and the observations made. Section \ref{sec-results} reports the concept and process models. Section \ref{sec-related} discusses the results and the connections with the existing literature. Section \ref{sec-threats} details the threats to validity and the related mitigation strategies. Finally, in Section \ref{sec-conclusions}, we present the conclusions and future work.