\begin{abstract}
\textbf{Context:} Experimental replication is complex in Software Engineering. Several strategies, such as replication packages and families of experiments, have been applied with limited success. We wonder whether the problems are due to formal issues, e.g., under-specification, miscommunication, etc., or intrinsic reasons, i.e., the researchers' needs are not fulfilled by the proposed replication procedures.
\textbf{Objective:} To find out how ESE researchers conduct experiments in practice.
\textbf{Method:} We carried out an ethnographical study with an experimental research group for two years instead of working halfway with several research groups over short periods.
\textbf{Results:} We have created conceptual and process models representing how experimentation, replication and synthesis are conducted in the target research group. These models fit the mainstream procedures and terminology at a high level but depart at lower, specialized levels. The experimental process carried out in the group differs from textbooks in terms of (1) the number and diversity of activities, (2) the existence of different roles, (3) the granularity of the concepts used by the roles, and (4) the viewpoints that different sub-areas or families of experiments have about the experimental process.
\textbf{Conclusions:} The differences between the actual lab processes vs the idealized versions portrayed in textbooks probably harm knowledge transfer among labs and make replication harder. However, our observations are limited to one research group. We plan to conduct further research with a representative sample of the research groups, e.g., using a survey, to explore whether our findings have general applicability.[CHECK NO MORE THAN 200 WORDS]
\end{abstract}