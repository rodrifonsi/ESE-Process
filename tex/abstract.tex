\begin{abstract}
\textbf{Context:} Replication is complex in Experimental Software Engineering (ESE). Replication packages and families of experiments have been applied with limited success. We wonder whether the problems are due to formal issues, e.g., under-specification, miscommunication, or intrinsic reasons, i.e., the existing replication procedures do not fulfill the researchers' needs.
\textbf{Objective:} Find out how ESE researchers conduct experiments in practice.
\textbf{Method:} Ethnographical study with an experimental research group.
\textbf{Results:} We have created conceptual and process models representing experimentation, replication, and synthesis in the target research group. These models fit the mainstream procedures and terminology at a high level but depart at lower, specialized levels. The experimental process carried out in the group differs from folk knowledge and textbooks in (1) the number and diversity of activities, (2) the existence of different roles, (3) the granularity of the concepts, and (4) the viewpoints that different sub-areas or families of experiments have about the experimental process.
\textbf{Conclusions:} The differences between the actual lab processes vs. the idealized versions probably harm knowledge transfer and make replication harder. We will conduct further research with more research groups, e.g., using a survey to explore whether our findings have general applicability.
\end{abstract}