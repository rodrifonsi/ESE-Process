\section{Threats to Validity}\label{sec-threats}
Validity is the degree to which the research is conducted transparently and without biases. We have followed Runeson \textit{et al.}'s recommendations \cite[p. 71–73]{Runenson-2012-case-study-SE} to design the ethnographic research and describe the threats to validity. Runeson \textit{et al.}, following Yin \cite{Yin-2009-case-study}, classify threats into four groups: (1) construct validity, (2) internal validity, (3) external validity, and (4) reliability.

\subsection{Construct validity}
Construct validity deals with the degree of agreement between the study constructs, i.e., what the ethnographic researchers have in mind, and the observations made in the field. For example, if the interview questions are not interpreted the same way by the researcher and the experimenters, there is a threat to construct validity. 

The ethnographic method is particularly suitable to oppose this threat because knowledge acquisition progresses gradually. On top of that, we used several techniques during the research to create constructs, e.g., reading, observation, and interviews, among others. The different approaches increase the chances of identifying or solving misunderstandings between the researchers and the RGUS. Using reliable information sources, e.g., standard literature and the RGUS' experimental material, we also prevented construct validity threats. 

Finally, the constructs were validated iteratively by experimenters and researchers during the research project lifespan.

\subsection{Internal validity}
Internal validity addresses the credibility of the causal relationships found during the research. However, this threat does not apply to this ethnographic study, as we do not aim to identify causal relationships but, at most, correlations between phenomena, e.g., when we claim that the number and diversity of experimental tasks explain the existence of roles. Furthermore, most of the findings we provide are descriptive, e.g., disagreements among experimenters of different experimental families.

\subsection{External Validity}
External validity represents the degree of certainty of the findings obtained in an investigation being generalizable, i.e., applied to different contexts of their origin. This threat is prominent in this study since it was carried out in a single instance (an experimental research group) of the population under study (SE research groups that apply empirical methods). As a countermeasure, we chose a highly representative experimental research group.

\subsection{Reliability}
The reliability of a study is the ease with which the research activities and the results obtained can be reproduced by other studies that apply the same methodology in the same RGUS (more likely, in similar experimental research groups). 

As mitigation measures, we have followed the research methodology outlined in Section~\ref{sec-research-method} closely to ensure the reliability of this research. Moreover, we have disclosed all intermediate and final results in a repository available under request to guarantee transparency in the investigation.