\section{Discussion and related work}\label{sec-related}
This research aims to find out how experimenters work in practice. After conducting the ethnographic research, we obtained models (process, workflow, and conceptual). The conceptual model connects with empirical researchers' "prominent picture" of their research area. The process and workflow models represent a novel contribution; to our knowledge, such detailed descriptions of the experimental process are unavailable in the existing literature.

The models described in Section~\ref{sec-results} are available in full detail online. In this section, we would like to focus on some interesting model features:

\begin{itemize}
	\item \textbf{The experimental process is broader} than we usually recognize as "experimentation". An experiment can be characterized as a small (or not that small) research project. It includes activities related to (see this \href{https://zenodo.org/record/7105096#.YyxsxezMLUJ}{\ul{link}}):
		\begin{itemize}
			\item management (resources, research lines, among others.),
			\item negotiation (e.g., with companies), 
			\item document/artifact support, 
			\item publication, among others.
		\end{itemize}
Although all empirical researchers know these activities, it is surprising that they are seldom mentioned (if ever) in ESE literature. To the best of our knowledge, reference texts, e.g., \cite{Wohlin-2000-Experimentation-SE}, \cite{Juristo-2001-SE-experimentation}, \textbf{do not address these activities beyond marginal notes}. Such richness in the number and diversity of activities raises obvious consequences in experimental education and training.
	\item The ethnographic study revealed one consequence of such diversity. It is possible, albeit unlikely, that one experimenter performs all experimental activities. These activities are not only complex: they demand time. In a project (research or not), the diversity of activities is usually associated with roles with well-defined responsibilities. \textbf{Also in experimentation. We identified three roles}: (1) \textit{Research Manager}, (2) \textit{Experiment Manager}, and (3) \textit{Senior Experimenter}. These roles are probably related to the RGUS' size; smaller (or larger) groups may have more generic/specialized roles. The concept of~\textquotedblleft role\textquotedblright~applied to experimentation is unusual but not entirely new. Mohamed et al. \cite{Mohamed-1993-roles-ESE} proposed a framework to conduct ESE experiments. This framework was based on a merge between statistical and SE process concepts, \textbf{including explicit roles}.
	\item The high-level conceptual model (see this \href{https://zenodo.org/record/7102405#.YyxvFOzMLUK}{\ul{link}}) could be endorsed by any experimenter (with some complaints depending on their specialization area, as it happened during the ethnography). However, when the concept of a role appears in the scene, the high-level model is just \textbf{an abstraction of the information handled by the roles}. In other words: the high-level model is incomplete from the roles' viewpoint and thus partially valid. The high-level model evokes the viewpoint of the \textit{senior experimenter} (see this \href{https://zenodo.org/record/7102464#.YyxvsezMLUK}{\ul{link}}), but with fewer details. In turn, the \textit{experiment manager} (see this \href{https://zenodo.org/record/7102450#.Yyxv6ezMLUL}{\ul{link}}), and the \textit{research manager} (see this \href{https://zenodo.org/record/7102431#.YyxwGezMLUL}{\ul{link}}) pay little attention to the design and execution details because the \textit{senior experimenter} takes care of that. Likewise, the \textit{experiment manager} is not affected by the experimental research in its global context, e.g., collaboration with other groups and replication. Such a concern fits the \textit{research manager}'s role. 
\item We deliberately built the models (process, workflow, and conceptual) to offer a coherent picture. However, as we indicated in Section~\ref{subsubsec-focus-groups}, we could not address the existence of different families of experiments. When experimenters working in other families get together, the models automatically diverge. Therefore, the models represent the maximum achievable consensus obtained during the role-oriented meetings. We are conscious that \textbf{further research is necessary at the family level to discover the peculiarities of each research area}. Such an inquiry could have an impact on the SE community. 
\item The replication problem is related to knowledge transfer \cite{Shull-2004-Knowledge-sharing-issues-SE}. Nevertheless, we wonder what exactly does it mean to transfer knowledge? To what extent or how should it be done? For example, Ferreira et al. \cite{Ferreira-2017-planning-experiments} emphasize methodological consistency over discussing methods and techniques. 

There is, apparently, an underlying dichotomy between theory and practice. In theory, formal guides, e.g., replication packages, should describe the process. However, the practice shows that such mechanisms cannot cope with the existing diversity of experimental protocols. The replication process will likely benefit from the specialized, fine-grained knowledge \textbf{at the family level}.
\end{itemize}

The models (process, workflow, and conceptual) and the ethnographic experience have generated some SE experimental process' differential characteristics or viewpoints, e.g., context adaptation and iterative improvement. Previous research pointed out some of these characteristics, e.g., \cite{Mohamed-1993-roles-ESE}, \cite{Sjoberg-2005-survey-experiments-SE}, \cite{Shull-2004-Knowledge-sharing-issues-SE}. We do not wish to provide a detailed account of these differential characteristics herein; that would require a publication of a different character. Instead, we aimed to observe how experimenters work and found out that they perform tasks other than those portrayed in textbooks. \textbf{This is our essential contribution}.

We also know that models represent \textbf{the point of view of a single experimental research group}. We must extend the inquiry to other experimental research groups, ideally to large portions of the ESE community, and check whether they reveal similar behavior patterns.